\newcommand{\HRule}{\rule{\linewidth}{0.5mm}} 

\begin{titlepage}
\center
%	HEADING & LOGO
\textsc{
\Huge \textbf{Department of Electronic \& Telecommunication Engineering}\\[.5cm]
\Large
University of Moratuwa\\[1.5cm] 
\includegraphics[width=5cm]{Picture1.png}\\[2cm]
EN1093 : Laboratory Practice I}\\[1.5cm] 


%	TITLE 
\sffamily
\HRule \\[0.4cm]
\textbf{\Huge\textcolor{deepskyblue}{Digital Alarm Clock}}\\[0.2cm] 
\HRule \\[1cm]
 \Large
Group 25\\[5mm]
%	AUTHORS & SUPERVISOR
\large
\begin{minipage}[t]{.5\textwidth}
\begin{flushleft}
\emph{ \Large Team Members}

\large 190622R \quad O.K.D. Tharindu\\
\large 190626H \quad P.D. Tharundi \\
\large 190631T \quad T.A. Thenuwara\\
\large 190636M \quad A.G.N. Udara\\
\end{flushleft}

\end{minipage}\hfill\begin{minipage}[t]{.4\textwidth}

\begin{flushright}
\emph{ \Large Course Instructor} 

Randima Senanayake\\ 

\end{flushright}
\end{minipage}
\end{titlepage}

\begin{abstract}
As the final assessment of the EN1093 module, we were given the task of designing a Digital alarm clock. Our Digital alarm clock is a standard Alarm clock with all necessary functionalities and including a simple Timer functionality. We were instructed to use AVR C++ for firmware development and, Proteus for simulations. Also, to use Atmega328P Microcontroller, 16x2 LCD display and for the RTC IC, DS1307 or DS3231. For sound output, we used a simple active piezo buzzer.
This document contains the process of creating our Alarm clock, issues we faced, challenges, functionalities, firmware details, and technical specifications.\\    
We also attached a 3D object of our Enclosure design, be sure to use Adobe reader DC for a good experience.
\end{abstract}
